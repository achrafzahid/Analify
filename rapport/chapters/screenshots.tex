\chapter{Captures d'écran et Démonstration de l'Interface}

Ce chapitre présente les principales interfaces de la plateforme Analify à travers des captures d'écran commentées. L'objectif est de donner une vision concrète de l'expérience utilisateur et de montrer comment les fonctionnalités décrites dans les chapitres précédents se matérialisent visuellement.

\section{Page d'Accueil et Landing Page}

\subsection{Landing Page}

La page d'accueil constitue le premier point de contact avec la plateforme. Elle présente de manière claire et attractive les principales fonctionnalités d'Analify.

\begin{figure}[H]
	\centering
	% TODO: Insérer la capture d'écran de la landing page
	\includegraphics[width=0.8\textwidth]{Images/1st.png}
	\caption{Landing Page - Interface d'accueil de la plateforme Analify}
	\label{fig:landing-page}
\end{figure}

\textbf{Éléments visibles :}
\begin{itemize}
	\item Logo et branding Analify
	\item Menu de navigation (Accueil, Fonctionnalités, À propos, Connexion)
	\item Section hero avec titre accrocheur et call-to-action
	\item Présentation des trois piliers : Analytics, Bidding, Assistant IA
	\item Footer avec informations de contact
\end{itemize}

\textbf{Technologies utilisées :}
\begin{itemize}
	\item React + TypeScript pour la structure
	\item Tailwind CSS pour le design responsive
	\item shadcn/ui pour les composants modernes
	\item Animations et transitions fluides
\end{itemize}

\section{Authentification et Connexion}

\subsection{Page de Connexion}

L'interface de connexion est sobre et sécurisée, permettant aux utilisateurs de s'authentifier avec leurs identifiants.

\begin{figure}[H]
	\centering
	% TODO: Insérer la capture d'écran de la page de login
	\includegraphics[width=0.8\textwidth]{Images/2nd.png}
	\caption{Page de Connexion - Authentification JWT}
	\label{fig:login-page}
\end{figure}

\textbf{Fonctionnalités :}
\begin{itemize}
	\item Formulaire de connexion avec validation côté client
	\item Champs email et mot de passe sécurisés
	\item Messages d'erreur clairs en cas d'échec d'authentification
	\item Redirection automatique vers le dashboard après connexion réussie
	\item Gestion des tokens JWT stockés de manière sécurisée
\end{itemize}

\textbf{Sécurité :}
\begin{itemize}
	\item Authentification basée sur JWT (JSON Web Token)
	\item Tokens expirés après 24 heures
	\item Protection CSRF et CORS configurée côté backend
	\item Hashage des mots de passe avec BCrypt
\end{itemize}

\section{Dashboard Principal - Vue d'Ensemble}

\subsection{Dashboard pour Administrateur Global}

Le tableau de bord principal offre une vue consolidée de l'ensemble des indicateurs métier, avec accès à toutes les statistiques de la plateforme.

\begin{figure}[H]
	\centering
	% TODO: Insérer la capture d'écran du dashboard admin global
	\includegraphics[width=0.8\textwidth]{Images/3rd.png}
	\caption{Dashboard Administrateur Global - Vue d'ensemble complète}
	\label{fig:dashboard-admin-global}
\end{figure}

\textbf{Indicateurs affichés (ADMIN\_G) :}
\begin{itemize}
	\item \textbf{Revenus totaux} : Agrégation de tous les magasins
	\item \textbf{Valeur totale du stock} : Inventaire global valorisé
	\item \textbf{Nombre de commandes} : Total des transactions
	\item \textbf{Produits vendus} : Quantité totale écoulée
	\item \textbf{Graphique d'évolution des revenus} : Courbe temporelle (line chart)
	\item \textbf{Top 10 magasins} : Classement par chiffre d'affaires (bar chart)
	\item \textbf{Top 10 produits} : Produits les plus vendus (bar chart)
	\item \textbf{Distribution par catégorie} : Répartition des ventes (pie chart)
	\item \textbf{Alertes stock faible} : Nombre de produits sous le seuil
\end{itemize}

\subsection{Dashboard pour Responsable de Magasin}

Les responsables de magasin (ADMIN\_STORE) ont accès uniquement aux données de leur(s) magasin(s) assigné(s).

\begin{figure}[H]
	\centering
	% TODO: Insérer la capture d'écran du dashboard admin store
	\includegraphics[width=0.8\textwidth]{Images/4th.png}

	\caption{Dashboard Responsable de Magasin - Vue limitée à son périmètre}
	\label{fig:dashboard-admin-store}
\end{figure}

\textbf{Périmètre restreint :}
\begin{itemize}
	\item Statistiques limitées aux magasins gérés par l'utilisateur
	\item Impossibilité de voir les données d'autres magasins
	\item Filtres pré-appliqués automatiquement par le backend
	\item KPI identiques mais calculés sur un sous-ensemble
\end{itemize}

\subsection{Dashboard pour Investisseur}

Les investisseurs (INVESTOR) visualisent uniquement les performances de leurs propres investissements et sections gagnées.

\begin{figure}[H]
	\centering
	% TODO: Insérer la capture d'écran du dashboard investisseur
	\includegraphics[width=0.8\textwidth]{Images/5th.png}
	\caption{Dashboard Investisseur - Suivi de portefeuille}
	\label{fig:dashboard-investor}
\end{figure}

\textbf{Indicateurs spécifiques :}
\begin{itemize}
	\item \textbf{Sections gagnées} : Nombre d'enchères remportées
	\item \textbf{Montant investi total} : Somme des bids gagnants
	\item \textbf{Revenus générés} : Performance des emplacements
	\item \textbf{ROI} : Retour sur investissement calculé
	\item \textbf{Graphiques de performance} : Évolution temporelle des gains
	\item \textbf{Alertes stock faible} : Pour les produits dans leurs sections
\end{itemize}

\section{Module d'Analytique Avancée}

\subsection{Statistiques Enrichies}

Le module d'analytique avancée offre des visualisations interactives et des métriques approfondies.

\begin{figure}[H]
	\centering
	% TODO: Insérer la capture d'écran des statistiques avancées
	\includegraphics[width=0.8\textwidth]{Images/6th.png}
	\caption{Statistiques Enrichies - Analytique avancée avec filtres}
	\label{fig:enhanced-stats}
\end{figure}

\textbf{Fonctionnalités avancées :}
\begin{itemize}
	\item \textbf{Filtres dynamiques} : Par date, magasin, produit, catégorie, investisseur
	\item \textbf{Graphiques interactifs} : Zoom, tooltip, sélection de périodes
	\item \textbf{Prédictions} : Tendances futures basées sur l'historique
	\item \textbf{Analyse comparative} : Comparaison de périodes (mois, trimestres)
	\item \textbf{Exports} : Téléchargement CSV et PDF des rapports
	\item \textbf{Tableaux détaillés} : Données brutes paginées et triables
\end{itemize}

\subsection{Panel de Filtres}

Le composant \texttt{FilterPanel} permet un contrôle fin des données affichées.

\begin{figure}[H]
	\centering
	% TODO: Insérer la capture d'écran du panel de filtres
	\includegraphics[width=0.8\textwidth]{Images/7th.png}
	\caption{Panel de Filtres - Contrôle granulaire des visualisations}
	\label{fig:filter-panel}
\end{figure}

\textbf{Contrôles disponibles :}
\begin{itemize}
	\item Sélecteur de plage de dates (date picker)
	\item Dropdown de sélection de magasin (autorisés uniquement)
	\item Dropdown de sélection de produit
	\item Dropdown de sélection d'investisseur (ADMIN\_G uniquement)
	\item Bouton \og Appliquer les filtres \fg{}
	\item Bouton \og Réinitialiser \fg{} pour revenir aux valeurs par défaut
\end{itemize}

\section{Module de Gestion des Produits}

\subsection{Liste des Produits}

L'interface de gestion des produits affiche l'inventaire complet avec possibilité de recherche, filtrage et tri.

\begin{figure}[H]
	\centering
	% TODO: Insérer la capture d'écran de la liste des produits
	\includegraphics[width=0.8\textwidth]{Images/prod.png}
	\caption{Liste des Produits - Gestion d'inventaire}
	\label{fig:products-list}
\end{figure}

\textbf{Informations affichées :}
\begin{itemize}
	\item Nom du produit
	\item Catégorie
	\item Prix unitaire
	\item Quantité en stock
	\item Seuil de stock minimum
	\item Statut (En stock / Stock faible / Rupture)
	\item Actions (Voir détails, Modifier, Supprimer)
\end{itemize}

\textbf{Fonctionnalités :}
\begin{itemize}
	\item Recherche en temps réel par nom ou catégorie
	\item Tri par colonne (nom, prix, stock)
	\item Pagination des résultats
	\item Badges visuels pour les alertes stock faible
	\item Export de la liste en CSV
\end{itemize}


\section{Module de Gestion des Commandes}

\subsection{Liste des Commandes}

L'interface de commandes permet de suivre l'ensemble des transactions de vente.

\begin{figure}[H]
	\centering
	% TODO: Insérer la capture d'écran de la liste des commandes
	\includegraphics[width=0.8\textwidth]{Images/order.png}
	\caption{Liste des Commandes - Suivi des ventes}
	\label{fig:orders-list}
\end{figure}

\textbf{Colonnes affichées :}
\begin{itemize}
	\item Numéro de commande
	\item Date et heure de création
	\item Montant total
	\item Nombre d'articles
	\item Client/Caissier
	\item Magasin
	\item Statut (Complétée, En cours, Annulée)
\end{itemize}

\textbf{Fonctionnalités :}
\begin{itemize}
	\item Filtrage par période, magasin, statut
	\item Recherche par numéro de commande
	\item Export des données en CSV
	\item Détails de chaque commande au clic
	\item Indicateurs de performance (commandes/jour, panier moyen)
\end{itemize}

\subsection{Création d'une Nouvelle Commande}

Le formulaire de création de commande permet aux caissiers d'enregistrer les ventes.

\begin{figure}[H]
	\centering
	% TODO: Insérer la capture d'écran du formulaire de création de commande
	\includegraphics[width=0.8\textwidth]{Images/addorder.png}
	\caption{Création de Commande - Interface caissier}
	\label{fig:create-order}
\end{figure}

\textbf{Étapes du processus :}
\begin{itemize}
	\item Sélection du magasin (pré-rempli pour CAISSIER)
	\item Ajout de produits via recherche/autocomplete
	\item Définition des quantités
	\item Calcul automatique du total avec taxes
	\item Validation de disponibilité du stock
	\item Confirmation et enregistrement
	\item Mise à jour automatique du stock
\end{itemize}

\section{Module de Bidding - Système d'Enchères}

\subsection{Navigation par Catégories}

Le système de bidding commence par la sélection d'une catégorie d'emplacement.

\begin{figure}[H]
	\centering
	% TODO: Insérer la capture d'écran de la navigation catégories
	\includegraphics[width=0.8\textwidth]{Images/investigate.png}
	\caption{Catégories de Bidding - Navigation hiérarchique}
	\label{fig:bidding-categories}
\end{figure}

\textbf{Organisation hiérarchique :}
\begin{enumerate}
	\item \textbf{Catégories} : Grandes familles d'emplacements (Électronique, Alimentaire, Mode, etc.)
	\item \textbf{Rangs} : Emplacements au sein d'une catégorie
	\item \textbf{Faces} : Côtés d'un rang (Face A, B, C, D)
	\item \textbf{Sections} : Subdivisions d'une face (unité minimale d'enchère)
\end{enumerate}

\subsection{Vue des Sections Disponibles}

L'interface affiche les sections ouvertes aux enchères avec leurs caractéristiques.

\begin{figure}[H]
	\centering
	% TODO: Insérer la capture d'écran des sections disponibles
	\includegraphics[width=0.8\textwidth]{Images/sec.png}
	\caption{Sections Disponibles - Emplacements ouverts au bidding}
	\label{fig:bidding-sections}
\end{figure}

\textbf{Informations par section :}
\begin{itemize}
	\item Nom et localisation (Catégorie > Rang > Face > Section)
	\item Statut (Ouverte, En cours, Fermée, Attribuée)
	\item Prix de réserve (montant minimum)
	\item Enchère actuelle la plus haute
	\item Date limite de clôture
	\item Nombre d'enchérisseurs
	\item Bouton \og Placer une enchère \fg{}
\end{itemize}

\subsection{Placement d'une Enchère}

Le formulaire de placement d'enchère permet aux investisseurs de soumettre leurs offres.

\begin{figure}[H]
	\centering
	% TODO: Insérer la capture d'écran du formulaire de bid
	\includegraphics[width=0.8\textwidth]{Images/sec2.png}
	\caption{Placement d'Enchère - Formulaire d'investissement}
	\label{fig:place-bid}
\end{figure}

\textbf{Processus de bidding :}
\begin{itemize}
	\item Affichage du prix de réserve et de l'enchère actuelle
	\item Saisie du montant proposé
	\item Validation automatique (montant > enchère actuelle)
	\item Message de confirmation
	\item Notification en cas de surenchère par un autre investisseur
	\item Mise à jour en temps réel de l'enchère gagnante
\end{itemize}

\subsection{Suivi des Enchères}

Les investisseurs peuvent consulter l'historique de leurs enchères.

\begin{figure}[H]
	\centering
	% TODO: Insérer la capture d'écran du suivi des bids
	\includegraphics[width=0.8\textwidth]{Images/suiv.png}
	\caption{Mes Enchères - Suivi du portefeuille d'investissements}
	\label{fig:my-bids}
\end{figure}

\textbf{Informations de suivi :}
\begin{itemize}
	\item Liste des sections sur lesquelles l'investisseur a enchéri
	\item Montant de chaque enchère
	\item Statut (Gagnant, Surenchéri, En cours, Fermée)
	\item Date de placement
	\item Historique complet des enchères par section
	\item Notifications de changement de statut
\end{itemize}

\section{Assistant Analytique LLM}

\subsection{Interface Chat}

L'assistant analytique offre une interface conversationnelle pour interroger les données en langage naturel.

\begin{figure}[H]
	\centering
	% TODO: Insérer la capture d'écran de l'assistant LLM
	\includegraphics[width=0.8\textwidth]{Images/llm1.png}
	\caption{Assistant Analytique LLM - Interface conversationnelle}
	\label{fig:llm-assistant}
\end{figure}

\textbf{Fonctionnalités de l'assistant :}
\begin{itemize}
	\item \textbf{Questions en langage naturel} : "Quels sont mes produits les plus rentables ce mois-ci ?"
	\item \textbf{Réponses contextualisées} : L'assistant utilise les données agrégées filtrées par rôle
	\item \textbf{Historique de conversation} : Maintien du contexte sur plusieurs échanges
	\item \textbf{Suggestions de questions} : Propositions basées sur le profil utilisateur
	\item \textbf{Visualisations intégrées} : L'assistant peut référencer les graphiques existants
	\item \textbf{Export des conversations} : Sauvegarde de l'historique
\end{itemize}

\textbf{Architecture technique :}
\begin{itemize}
	\item Modèle : LLaMA 3.2:3b via Ollama (local, sans limite de taux)
	\item Backend : Spring AI pour l'intégration
	\item Contexte : 8192 tokens (conversations étendues)
	\item Accélération : GPU RTX 4060 (100\% GPU)
	\item Temps de réponse : 200-500ms (modèle en mémoire)
\end{itemize}

\subsection{Exemples de Questions}

L'assistant peut répondre à divers types de questions métier :

\begin{figure}[H]
	\centering
	% TODO: Insérer la capture d'écran d'exemples de questions
	\includegraphics[width=0.8\textwidth]{Images/llm2.png}
	\caption{Exemples de Questions - Suggestions contextuelles}
	\label{fig:example-questions}
\end{figure}

\textbf{Catégories de questions supportées :}
\begin{itemize}
	\item \textbf{KPI et performance} : "Quel est mon chiffre d'affaires ce mois-ci ?"
	\item \textbf{Produits} : "Quels produits sont en rupture de stock ?"
	\item \textbf{Magasins} : "Quel magasin performe le mieux ?"
	\item \textbf{Tendances} : "Les ventes sont-elles en hausse ou en baisse ?"
	\item \textbf{Investissements} : "Combien ai-je investi dans le bidding ce mois-ci ?"
	\item \textbf{Comparaisons} : "Comment se compare mon magasin par rapport à la moyenne ?"
\end{itemize}

\section{Gestion des Utilisateurs et Rôles}

\subsection{Liste des Utilisateurs (Admin)}

Les administrateurs globaux peuvent gérer l'ensemble des comptes utilisateurs.

\begin{figure}[H]
	\centering
	% TODO: Insérer la capture d'écran de la gestion des utilisateurs
	\includegraphics[width=0.8\textwidth]{Images/users.png}
	\caption{Gestion des Utilisateurs - Administration des comptes}
	\label{fig:users-management}
\end{figure}

\textbf{Fonctionnalités d'administration :}
\begin{itemize}
	\item Création de nouveaux utilisateurs
	\item Attribution des rôles (ADMIN\_G, ADMIN\_STORE, INVESTOR, CAISSIER)
	\item Assignation de magasins aux responsables
	\item Modification des permissions
	\item Désactivation/Suppression de comptes
	\item Réinitialisation de mots de passe
\end{itemize}

\subsection{Profil Utilisateur}

Chaque utilisateur peut consulter et modifier ses informations personnelles.

\begin{figure}[H]
	\centering
	% TODO: Insérer la capture d'écran du profil utilisateur
	\includegraphics[width=0.8\textwidth]{Images/profile.png}
	\caption{Profil Utilisateur - Informations personnelles}
	\label{fig:user-profile}
\end{figure}

\textbf{Informations modifiables :}
\begin{itemize}
	\item Nom et prénom
	\item Email (vérification requise)
	\item Téléphone
	\item Adresse
	\item Photo de profil
	\item Changement de mot de passe
	\item Préférences de notification
\end{itemize}

\section{Exports et Rapports}

\subsection{Génération de Rapports PDF}

Les utilisateurs peuvent générer des rapports PDF de leurs statistiques.

\begin{figure}[H]
	\centering
	% TODO: Insérer la capture d'écran du rapport PDF généré
	\includegraphics[width=0.8\textwidth]{Images/ex.png}
	\caption{Rapport PDF - Export professionnel}
	\label{fig:pdf-report}
\end{figure}

\textbf{Contenu du rapport :}
\begin{itemize}
	\item En-tête avec logo et période
	\item KPI principaux en tableau
	\item Graphiques exportés en images
	\item Tableaux de données détaillés
	\item Pied de page avec date de génération
	\item Mise en page professionnelle
\end{itemize}

\subsection{Export CSV}

Les données brutes peuvent être exportées au format CSV pour analyse externe.

\begin{figure}[H]
	\centering
	% TODO: Insérer la capture d'écran du bouton d'export CSV
	\includegraphics[width=0.8\textwidth]{Images/expo.png}
	\caption{Export CSV - Données brutes pour tableur}
	\label{fig:csv-export}
\end{figure}

\textbf{Données exportables :}
\begin{itemize}
	\item Liste complète des produits
	\item Historique des commandes
	\item Statistiques de vente par période
	\item Inventaire avec niveaux de stock
	\item Historique des enchères
	\item Performance des investissements
\end{itemize}

\section{Messages d'Erreur et Notifications}

\subsection{Gestion des Erreurs}

L'application affiche des messages d'erreur clairs et exploitables.

\begin{figure}[H]
	\centering
	% TODO: Insérer la capture d'écran d'un message d'erreur
	\includegraphics[width=0.8\textwidth]{Images/err.png}
	\caption{Messages d'Erreur - Retour utilisateur explicite}
	\label{fig:error-messages}
\end{figure}

\textbf{Types de messages :}
\begin{itemize}
	\item Erreurs de validation (champs manquants, formats invalides)
	\item Erreurs de permissions (accès refusé)
	\item Erreurs serveur (500, connexion perdue)
	\item Avertissements (stock faible, date limite proche)
	\item Succès (commande créée, bid placé)
\end{itemize}

\subsection{Système de Notifications}

Les utilisateurs reçoivent des notifications pour les événements importants.

\begin{figure}[H]
	\centering
	% TODO: Insérer la capture d'écran du système de notifications
	\includegraphics[width=0.8\textwidth]{Images/not.png}
	\caption{Système de Notifications - Alertes temps réel}
	\label{fig:notifications}
\end{figure}

\textbf{Événements notifiés :}
\begin{itemize}
	\item Enchère surenchérie
	\item Section remportée
	\item Clôture d'enchère
	\item Stock critique atteint
	\item Nouvelle commande (pour admin)
	\item Connexion réussie/échouée
	\item Export terminé
\end{itemize}

\section{Conclusion du Chapitre}

Ce chapitre a présenté de manière visuelle les principales interfaces de la plateforme Analify. Les captures d'écran illustrent concrètement :

\begin{itemize}
	\item La cohérence visuelle et l'ergonomie de l'application
	\item L'adaptation des interfaces selon les rôles utilisateurs
	\item La richesse fonctionnelle (analytics, bidding, assistant IA)
	\item La qualité du design responsive (desktop et mobile)
	\item L'attention portée à l'expérience utilisateur (UX)
\end{itemize}

L'ensemble de ces interfaces démontre la maturité du projet et sa capacité à répondre aux besoins métier identifiés dans les chapitres précédents. Le prochain chapitre conclura le rapport et proposera des perspectives d'évolution pour la plateforme.
